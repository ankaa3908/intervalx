%%
%% This is file `intervalx-doc-fr.tex',
%% generated with the docstrip utility.
%%
%% The original source files were:
%%
%% intervalx.dtx  (with options: `doc-fr')
%% 
%% This is a generated file.
%% 
%% Copyright (C) 2025 Valentin Dao
%% 
%% This file may be distributed and/or modified under the
%% conditions of the LaTeX Project Public License, either
%% version 1.3 of this license or (at your option) any later
%% version. The latest version of this license is in:
%% 
%%   http://www.latex-project.org/lppl.txt
%% 
%% and version 1.3c or later is part of all distributions of
%% LaTeX version 2008-05-04 or later.
%% 
%% This work has the LPPL maintenance status 'maintained'
%% and the current maintainer is Valentin Dao (vdao.texdev@gmail.com).
%% 
%% This work consists of the files intervalx.ins and intervalx.dtx,
%% along with the derived files intervalx-doc-en.pdf, and
%% intervalx-doc-fr.pdf. In order to acquire the .tex master files
%% producing the documentations and the .sty file, simply run the
%% installation file through any engine.
%% 
%% The repository of this work can be found at:
%% 
%%   https://github.com/ankaa3908/intervalx
%% 
\documentclass[full, check=true, 12pt]{l3doc}

\ExplSyntaxOn

\tl_new:N \l__intervalx_doc_update_tl
\tl_new:N \l__intervalx_doc_new_tl

\keys_define:nn { l3doc/function }
  {
    maj .tl_set:N = \l__intervalx_doc_update_tl,
    new .tl_set:N = \l__intervalx_doc_new_tl,
  }

\cs_set:Npn \__codedoc_typeset_dates:
  {
    \bool_lazy_all:nF
      {
        { \tl_if_empty_p:N \l__codedoc_date_added_tl }
        { \tl_if_empty_p:N \l__codedoc_date_updated_tl }
        { \tl_if_empty_p:N \l__intervalx_doc_update_tl }
        { \tl_if_empty_p:N \l__intervalx_doc_new_tl }
      }
      { \midrule }
    \tl_if_empty:NF \l__codedoc_date_added_tl
      {
        \multicolumn { 2 } { @{} r @{} }
          { \scriptsize New: \, \l__codedoc_date_added_tl } \\
      }
    \tl_if_empty:NF \l__codedoc_date_updated_tl
      {
        \multicolumn { 2 } { @{} r @{} }
          { \scriptsize Updated: \, \l__codedoc_date_updated_tl } \\
      }
    \tl_if_empty:NF \l__intervalx_doc_update_tl
      {
        \multicolumn{ 2 }{ @{} r @{} }
          {
            \itshape\sffamily\scriptsize
            Mis~à~jour:
            \, \l__intervalx_doc_update_tl
          } \\
      }
    \tl_if_empty:NF \l__intervalx_doc_new_tl
      {
        \multicolumn{ 2 }{ @{} r @{} }
          {
            \itshape\sffamily\scriptsize
            Nouveau:
            \, \l__intervalx_doc_new_tl
          } \\
      }
  }

\cs_undefine:N \tsremark
\cs_undefine:N \endtsremark
\ExplSyntaxOff

\usepackage{amsmath, amssymb}

\usepackage{fontspec}
\usepackage{unicode-math}

\setmainfont{Mercury-TextG4}[
  Extension              = .otf,
  UprightFont            = *-Roman-OK,
  UprightFeatures        = { Kerning = On },
  ItalicFont             = *-Italic-Pro-OK,
  BoldFont               = *Bold,
  BoldFeatures           = { SmallCapsFont = *SemiboldSC },
  BoldItalicFont         = *BoldItalic,
  SmallCapsFont          = *RomanSC,
]

\setsansfont{Whitney}[
  Extension       = .otf,
  UprightFont     = *-Medium,
  UprightFeatures = {Kerning = On, StylisticSet = 1},
  BoldFont        = *-Semibold,
  BoldFeatures    = {Kerning = On, StylisticSet = 1},
  ItalicFont      = *-MediumItalic,
  BoldItalicFont  = *-SemiboldItalic
]

\setmathfont{NewCMMath-Regular.otf}

\usepackage{intervalx}

\usepackage{microtype}
\usepackage{csquotes}
\usepackage{fontawesome5}
\usepackage{polyglossia}
\setmainlanguage{french}

\usepackage{titlesec}
\titleformat*{\section}{\sffamily\LARGE\bfseries}
\titleformat*{\subsection}{\sffamily\Large\bfseries}

\usepackage{luacode}

\begin{luacode}
function fontcopyright(fontname)
  local font = fontloader.open(kpse.find_file(fontname, 'opentype fonts'))
  if font then
    local metrics = fontloader.to_table(font)
    local copyright = metrics.copyright:gsub("https?://%S+$", "")
    local copyright = copyright:gsub("Copyright%s%(C%)", "\\copyright\\")
    local copyright = copyright:gsub("&", "\\&")
    tex.print('\\textit{' .. metrics.fullname .. '}\\par')
    tex.print(copyright)
  end
end
\end{luacode}

\usepackage{xcolor}
\definecolor{RoyalBlue}{RGB}{0, 35, 102}
\definecolor{RoyalRed}{RGB}{157, 16, 45}
\definecolor{RoyalGrey}{HTML}{DEDEDE}

\usepackage{codedescribe}
\setcodekeys{numbers=left, codeprefix={Code \LaTeX}, resultprefix={Résultat}}
\usepackage{changelog}
\usepackage{enumitem}
\renewenvironment{changelogitemize}
  {\begin{itemize}[label=\raisebox{0.5ex}{$\scriptscriptstyle\blacktriangleright$}]}
  {\end{itemize}}

\DeclareTranslation{French}{changelog}{Historique des modifications}

\setlength{\parindent}{0pt}

\usepackage{hyperref}
\hypersetup{
  pdfauthor   = {Valentin Dao},
  pdftitle    = {L'extension intervalx},
  pdfcreator  = {LuaLaTeX with hyperref package},
  linkcolor   = RoyalBlue,
  urlcolor    = RoyalRed
}

\usepackage{cleveref}

\def\sarg{*}
\def\filecreationdate{16-11-2025}
\def\default#1{\frenchspacing\hfill(défaut: #1)}

\IndexPrologue{
  \section*{Index}
  \addcontentsline{toc}{section}{Index}
}

\EnableCrossrefs
\DisableImplementation

\begin{document}

\DeleteShortVerb\|

\GetFileInfo{intervalx.sty}

\ExplSyntaxOn
\seq_gset_split:Nne \g_tmpa_seq { - } { \filedate }
\seq_reverse:N \g_tmpa_seq
\xdef\filedate{ \seq_use:Nn \g_tmpa_seq { - } }
\ExplSyntaxOff

\title{^^A
  L'extension \textsf{intervalx}^^A
  \thanks{Ce fichier décrit la version \fileversion}
}

\author{^^A
  Valentin Dao^^A
  \thanks{E-mail: \href{mailto:vdao.texdev@gmail.com}{vdao.texdev@gmail.com}}
}

\date{Publiée le \filecreationdate}

\maketitle

\begin{documentation}

\begin{abstract}

Le présent package a pour but d'étendre les fonctionnalités d'\pkg{interval}\footnote{\href{https://ctan.org/pkg/interval}{Voir sur \textsc{ctan}}} en améliorant la macro principale et en y ajoutant quelques-unes. Bien que l'implémentation soit modernisée car utilisant \pkg{expl3}, l'approche n'est que très peu différente de celle employée par Lars Madsen. Ainsi, la plupart des clés porteront le même nom. Même si \pkg{intervalx} se présente comme une extension d'\pkg{interval}, de nouvelles macros pour la composition d'inégalités ont tout de même été implémentées.

\end{abstract}

\tableofcontents

\subsection*{Licence}

\copyright\ 2025 Valentin Dao, publié sous la \LaTeX\ Project Public License (\textsc{lppl}) 1.3c

\subsection*{Polices}

\directlua{fontcopyright('Mercury-TextG4-Roman-OK.otf')}

\vspace*{\baselineskip}

\begingroup
\sffamily
\directlua{fontcopyright('Whitney-Medium.otf')}
\endgroup

\subsection*{Dépôt}

\href{https://github.com/ankaa3908/intervalx/tree/main}{\faGithub\ Voir dépôt GitHub}.

\section{Options de package}

Voici les options de package pouvant être déclarées avec \cs{usepackage}:

\begin{function}{soft fences}
  \begin{syntax}
  \meta{booléen}\default{false}
  \end{syntax}
  Cette clé remplace les crochets ouverts par des parenthèses.
\end{function}

\begin{function}{smart fences}
  \begin{syntax}
  \meta{booléen}\default{true}
  \end{syntax}
  Cette clé permet d'adapter automatiquement le sens des crochets à $+\infty$ et $-\infty$.
\end{function}

\begin{function}{separator}
  \begin{syntax}
  \meta{token}\default{,}
  \end{syntax}
  Contrôle le caractère inséré entre les deux bornes de l'intervalle.
\end{function}

\section{Composer un intervalle}

\begin{function}{\interval}
  \begin{syntax}
  \sarg\oarg{clés}\marg{liste de bornes}
  \end{syntax}
  Pour composer un intervalle, il suffit d'employer la macro en mode mathématique en renseignant les bornes sous forme d'une liste:
\end{function}

\begin{codestore}[ex1]
  \begin{equation}
    \interval{a, b}
  \end{equation}
\end{codestore}
\tsdemo{ex1}

\begin{function}{open, open right, open left}
  Pour changer le sens des crochets, on peut employer les options ici présentes.
\end{function}

\begin{codestore}[ex2]
  \begin{gather}
    \interval[open right]{a, b}\\
    \interval[open left]{a, b}\\
    \interval[open]{a, b}
  \end{gather}
\end{codestore}
\tsdemo{ex2}

Le sens des crochets s'adapte aussi de lui-même à \verb|-\infty| et \verb|+\infty|, pourvu que l'option associée demeure activée.

\begin{codestore}[ex3]
  \begin{equation}
    \interval{-\infty, +\infty}
  \end{equation}
\end{codestore}
\tsdemo{ex3}

\begin{function}{scaled}
  \begin{syntax}
  big, Big, bigg, Bigg, auto\default{auto}
  \end{syntax}
  Il est également possible d'ajuster la taille des crochets/parenthèses à travers la clé \texttt{scaled}, qui a le même comportement que dans le package \pkg{interval}: \texttt{auto} applique \cs{left} et \cs{right} tandis que les autres options utilisent les délimiteurs \cs{bigl}/\cs{bigr}\dots
\end{function}

\begin{codestore}[ex4]
  \begin{equation}
    \interval[scaled]{\frac{1}{2}, \frac{3}{2}}
  \end{equation}
\end{codestore}
\tsdemo{ex4}

Enfin, la variante étoilée permet de composer des intervalles d'entiers, en utilisant le package \pkg{stmaryrd}.\footnote{\href{https://ctan.org/pkg/stmaryrd}{Voir sur \textsc{ctan}}} Toutes les clés citées ci-dessus sont compatibles avec ces symboles.

\begin{codestore}[ex5]
  \begin{equation}
    \interval*{2, 10}
  \end{equation}
\end{codestore}
\tsdemo{ex5}

\section{Produit, réunion et intersection d'intervalles}

\pkg{intervalx} permet aussi de plus facilement saisir les relations de produit, de réunion et d'intersection d'intervalles.

\subsection{Produit}

\begin{function}{\xinterval}
  \begin{syntax}
  \oarg{clés}\marg{*-liste}
  \end{syntax}
  Pour les produits, l'argument principal prend la même forme qu'avec \cs{interval}, mais les différents intervalles sont séparés d'un astérisque.
\end{function}

\begin{tsremark}[Astuce mnémotechnique:]
  Le \enquote{x} au début du nom de la macro évoque le symbole du produit $\times$, tandis que l'astérisque est une façon en programmation de le dénoter.
\end{tsremark}

\begin{codestore}[ex6]
  \begin{equation}
    \xinterval{2, 10 * 1, 15 * -3, 19}
  \end{equation}
\end{codestore}
\tsdemo{ex6}

Pour combiner cette macro avec les clés d'\cs{interval}, ces dernières doivent être spécifiées sous forme d'une liste délimitée par le point-virgule (la virgule étant déjà prise pour séparer les différentes clés). Voici un exemple pour clarifier ce point.

\begin{codestore}[ex7]
  \begin{equation}
    \xinterval[open right, scaled; open left, scaled]{2, 10 * 1, 15}
  \end{equation}
\end{codestore}
\tsdemo{ex7}

\subsection{Réunion}

\begin{function}{\uinterval}
  \begin{syntax}
    \oarg{clés}\marg{\textbar-liste}
  \end{syntax}
  De même, la réunion d'intervalles se compose en utilisant cette fois-ci la barre verticale comme délimiteur.
\end{function}

\begin{tsremark}[Astuce mnémotechnique:]
  Le \enquote{u} au début du nom de la macro évoque le symbole de réunion $\cup$, tandis que la barre verticale est une façon en programmation de dénoter le \textit{ou} logique.
\end{tsremark}

\begin{codestore}[ex8]
  \begin{equation}
    \uinterval{2, 10 | 1, 15}
  \end{equation}
\end{codestore}
\tsdemo{ex8}

\subsection{Intersection}

\begin{function}{\ninterval}
  \begin{syntax}
    \oarg{clés}\marg{\&-liste}
  \end{syntax}
  De la même manière, les intersections d'intervalles se réalisent en utilisant l'éperluette comme délimiteur. L'emploi des clés est, là aussi, identique à \cs{xinterval} et \cs{uinterval}
\end{function}

\begin{tsremark}[Astuce mnémotechnique:]
  Le \enquote{n} évoque quant à lui le symbole d'intersection $\cap$, avec l'éperluette désignant en programmation le \emph{et} logique.
\end{tsremark}

\begin{codestore}[ex9]
  \begin{equation}
    \ninterval{3, 20 & \pi, e^3}
  \end{equation}
\end{codestore}
\tsdemo{ex9}

\section{Inégalités}

\begin{function}{\ineq}
  \begin{syntax}
    \sarg\oarg{clés}\marg{liste de bornes}\oarg{variable}
  \end{syntax}
  La composition d'inégalités est assez similaire à celle des intervalles, à quelques différences près. Les clés \texttt{open right}, \texttt{open left} et \texttt{open} sont aussi accessibles pour permettre de facilement écrire des inégalités strictes ou larges. En plus des bornes, il est également possible d'indiquer la variable, correspondant à $x$ par défaut. Enfin, la variante étoilée de la macro emploie les symboles alternatifs d'\pkg{amssymb} pour les inégalités larges.
\end{function}

\begin{codestore}[ex10]
  \begin{gather}
    \ineq{a, b}\\
    \ineq*{a, b}[y]\\
    \ineq[open right]{a, b}\\
    \ineq[open left]{a, b}\\
    \ineq[open]{a, b}
  \end{gather}
\end{codestore}
\tsdemo{ex10}

\addcontentsline{toc}{section}{Historique des modifications}

\begin{changelog}[sectioncmd=\section*]
\shortversion{v=1.0.0, date=16-11-2025, changes=Version initiale.}
\end{changelog}

\PrintIndex

\end{documentation}

\end{document}
\endinput
%%
%% End of file `intervalx-doc-fr.tex'.
