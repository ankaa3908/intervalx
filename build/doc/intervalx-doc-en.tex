%%
%% This is file `intervalx-doc-en.tex',
%% generated with the docstrip utility.
%%
%% The original source files were:
%%
%% intervalx.dtx  (with options: `doc-en')
%% 
%% This is a generated file.
%% 
%% Copyright (C) 2025 Valentin Dao
%% 
%% This file may be distributed and/or modified under the
%% conditions of the LaTeX Project Public License, either
%% version 1.3 of this license or (at your option) any later
%% version. The latest version of this license is in:
%% 
%%   http://www.latex-project.org/lppl.txt
%% 
%% and version 1.3c or later is part of all distributions of
%% LaTeX version 2008-05-04 or later.
%% 
%% This work has the LPPL maintenance status 'maintained'
%% and the current maintainer is Valentin Dao (vdao.texdev@gmail.com).
%% 
%% This work consists of the files intervalx.ins and intervalx.dtx,
%% along with the derived files intervalx-doc-en.pdf, and
%% intervalx-doc-fr.pdf. In order to acquire the .tex master files
%% producing the documentations and the .sty file, simply run the
%% installation file through any engine.
%% 
%% The repository of this work can be found at:
%% 
%%   https://github.com/ankaa3908/intervalx
%% 
\documentclass[full, check=true, 12pt]{l3doc}

\ExplSyntaxOn

\tl_new:N \l__intervalx_doc_update_tl
\tl_new:N \l__intervalx_doc_new_tl

\keys_define:nn { l3doc/function }
  {
    maj .tl_set:N = \l__intervalx_doc_update_tl,
    new .tl_set:N = \l__intervalx_doc_new_tl,
  }

\cs_set:Npn \__codedoc_typeset_dates:
  {
    \bool_lazy_all:nF
      {
        { \tl_if_empty_p:N \l__codedoc_date_added_tl }
        { \tl_if_empty_p:N \l__codedoc_date_updated_tl }
        { \tl_if_empty_p:N \l__intervalx_doc_update_tl }
        { \tl_if_empty_p:N \l__intervalx_doc_new_tl }
      }
      { \midrule }
    \tl_if_empty:NF \l__codedoc_date_added_tl
      {
        \multicolumn { 2 } { @{} r @{} }
          { \scriptsize New: \, \l__codedoc_date_added_tl } \\
      }
    \tl_if_empty:NF \l__codedoc_date_updated_tl
      {
        \multicolumn { 2 } { @{} r @{} }
          { \scriptsize Updated: \, \l__codedoc_date_updated_tl } \\
      }
    \tl_if_empty:NF \l__intervalx_doc_update_tl
      {
        \multicolumn{ 2 }{ @{} r @{} }
          {
            \itshape\sffamily\scriptsize
            Updated:
            \, \l__intervalx_doc_update_tl
          } \\
      }
    \tl_if_empty:NF \l__intervalx_doc_new_tl
      {
        \multicolumn{ 2 }{ @{} r @{} }
          {
            \itshape\sffamily\scriptsize
            New:
            \, \l__intervalx_doc_new_tl
          } \\
      }
  }

\cs_undefine:N \tsremark
\cs_undefine:N \endtsremark
\ExplSyntaxOff

\usepackage{amsmath, amssymb}

\usepackage{fontspec}
\usepackage{unicode-math}

\setmainfont{Mercury-TextG4}[
  Extension              = .otf,
  UprightFont            = *-Roman-OK,
  UprightFeatures        = { Kerning = On },
  ItalicFont             = *-Italic-Pro-OK,
  BoldFont               = *Bold,
  BoldFeatures           = { SmallCapsFont = *SemiboldSC },
  BoldItalicFont         = *BoldItalic,
  SmallCapsFont          = *RomanSC,
]

\setsansfont{Whitney}[
  Extension       = .otf,
  UprightFont     = *-Medium,
  UprightFeatures = {Kerning = On, StylisticSet = 1},
  BoldFont        = *-Semibold,
  BoldFeatures    = {Kerning = On, StylisticSet = 1},
  ItalicFont      = *-MediumItalic,
  BoldItalicFont  = *-SemiboldItalic
]

\setmathfont{NewCMMath-Regular.otf}

\usepackage{intervalx}

\usepackage{microtype}
\usepackage{csquotes}
\usepackage{fontawesome5}
\usepackage{polyglossia}
\setmainlanguage{english}

\usepackage{titlesec}
\titleformat*{\section}{\sffamily\LARGE\bfseries}
\titleformat*{\subsection}{\sffamily\Large\bfseries}

\usepackage{luacode}

\begin{luacode}
function fontcopyright(fontname)
  local font = fontloader.open(kpse.find_file(fontname, 'opentype fonts'))
  if font then
    local metrics = fontloader.to_table(font)
    local copyright = metrics.copyright:gsub("https?://%S+$", "")
    local copyright = copyright:gsub("Copyright%s%(C%)", "\\copyright\\")
    local copyright = copyright:gsub("&", "\\&")
    tex.print('\\textit{' .. metrics.fullname .. '}\\par')
    tex.print(copyright)
  end
end
\end{luacode}

\usepackage{xcolor}
\definecolor{RoyalBlue}{RGB}{0, 35, 102}
\definecolor{RoyalRed}{RGB}{157, 16, 45}
\definecolor{RoyalGrey}{HTML}{DEDEDE}

\usepackage{codedescribe}
\setcodekeys{numbers=left, codeprefix={\LaTeX code}, resultprefix={Result}}
\usepackage{changelog}
\usepackage{enumitem}
\renewenvironment{changelogitemize}
  {\begin{itemize}[label=\raisebox{0.5ex}{$\scriptscriptstyle\blacktriangleright$}]}
  {\end{itemize}}

\setlength{\parindent}{0pt}

\usepackage{hyperref}
\hypersetup{
  pdfauthor   = {Valentin Dao},
  pdftitle    = {The intervalx package},
  pdfcreator  = {LuaLaTeX with hyperref package},
  linkcolor   = RoyalBlue,
  urlcolor    = RoyalRed
}

\usepackage{cleveref}

\def\sarg{*}
\def\filecreationdate{2025-11-16}
\def\default#1{\frenchspacing\hfill(default: #1)}

\IndexPrologue{
  \section*{Index}
  \addcontentsline{toc}{section}{Index}
}

\EnableCrossrefs
\DisableImplementation

\begin{document}

\DeleteShortVerb\|

\GetFileInfo{intervalx.sty}

\title{^^A
  The \textsf{intervalx} package^^A
  \thanks{This file describes \fileversion}
}

\author{^^A
  Valentin Dao^^A
  \thanks{E-mail: \href{mailto:vdao.texdev@gmail.com}{vdao.texdev@gmail.com}}
}

\date{Released \filecreationdate}

\maketitle

\begin{documentation}

\begin{abstract}
The purpose of this package is to extend \pkg{interval}'s\footnote{\href{https://ctan.org/pkg/interval}{See on \textsc{ctan}.}} functionlities by improving the main macro and adding a few new ones. Although the implementation is has been modernised through the use of \pkg{expl3}, the approach is very similar to that used by Lars Madsen. As such, most keys will have the same name. Even though \pkg{intervalx} is presented as an extension of \pkg{interval}, new macros for composing inequalities have nevertheless been implemented.
\end{abstract}

\tableofcontents

\subsection*{License}

\copyright\ 2025 Valentin Dao, published under the \LaTeX\ Project Public License (\textsc{lppl}) 1.3c

\subsection*{Fonts}

\directlua{fontcopyright('Mercury-TextG4-Roman-OK.otf')}

\vspace*{\baselineskip}

\begingroup
\sffamily
\directlua{fontcopyright('Whitney-Medium.otf')}
\endgroup

\subsection*{Repository}

\href{https://github.com/ankaa3908/intervalx/tree/main}{\faGithub\ See GitHub repository}.

\section{Package options}

Here are the options that can be declared using \cs{usepackage}:

\begin{function}{soft fences}
  \begin{syntax}
  \meta{booléen}\default{false}
  \end{syntax}
  This key replaces open brackets with parentheses.
\end{function}

\begin{function}{smart fences}
  \begin{syntax}
  \meta{booléen}\default{true}
  \end{syntax}
  This key automatically adapts the direction of the brackets to the presence of $+\infty$ and $-\infty$
\end{function}

\begin{function}{separator}
  \begin{syntax}
  \meta{token}\default{,}
  \end{syntax}
  Controls the character inserted between the interval endpoints.
\end{function}

\section{Typeset an interval}

\begin{function}{\interval}
  \begin{syntax}
  \sarg\oarg{keys}\marg{list of endpoints}
  \end{syntax}
  To typeset an interval, simply use the macro in math mode, entering the endpoints in the form of a list:
\end{function}

\begin{codestore}[ex1]
  \begin{equation}
    \interval{a, b}
  \end{equation}
\end{codestore}
\tsdemo{ex1}

\begin{function}{open, open right, open left}
  To change the direction of the brackets, use the option provided here.
\end{function}

\begin{codestore}[ex2]
  \begin{gather}
    \interval[open right]{a, b}\\
    \interval[open left]{a, b}\\
    \interval[open]{a, b}
  \end{gather}
\end{codestore}
\tsdemo{ex2}

The direction of the brackets also adapts itself to \verb|-\infty| and \verb|+\infty|, provided that the corresponding option remains enabled.

\begin{codestore}[ex3]
  \begin{equation}
    \interval{-\infty, +\infty}
  \end{equation}
\end{codestore}
\tsdemo{ex3}

\begin{function}{scaled}
  \begin{syntax}
  big, Big, bigg, Bigg
  \end{syntax}
  It is also possible to adjust the brackets/parentheses' size through the \texttt{scaled} key, which has the same usage as in the \pkg{interval} package.
\end{function}

\begin{codestore}[ex4]
  \begin{equation}
    \interval[scaled=bigg]{\frac{1}{2}, \frac{3}{2}}
  \end{equation}
\end{codestore}
\tsdemo{ex4}

The starred variant typesets integer intervals by using the \pkg{stmaryrd} package.\footnote{\href{https://ctan.org/pkg/stmaryrd}{See on \textsc{ctan}}} All the keys described above are compatible with these symbols.

\begin{codestore}[ex5]
  \begin{equation}
    \interval*{2, 10}
  \end{equation}
\end{codestore}
\tsdemo{ex5}

\begin{function}{\ointerval, \linterval, \rinterval}
  \begin{syntax}
    \sarg\marg{open, keys}\marg{endpoints}
    \sarg\marg{open left, keys}\marg{endpoints}
    \sarg\marg{open right, keys}\marg{endpoints}
  \end{syntax}
  Finally, the package provides short hands through these three macros:
\end{function}

\begin{codestore}[ex6]
  \begin{gather}
    \ointerval[scaled=bigg]{\frac{1}{2}, \frac{9}{2}}\\
    \linterval{3, 8}\\
    \rinterval{3, 8}
  \end{gather}
\end{codestore}
\tsdemo{ex6}

\section{Interval product, union, and intersection}

\pkg{intervalx} also makes it easier to typeset product, union, and intersection relations between intervals.

\subsection{Product}

\begin{function}{\xinterval}
  \begin{syntax}
  \oarg{keys}\marg{*-list}
  \end{syntax}
  For the product, the main argument takes the same form as with \cs{interval}, but the different intervals are delimited by an asterisk.
\end{function}

\begin{tsremark}[Mnemonic:]
  The \enquote{x} at the beginning of the macro name evokes the product symbol $\times$, while the asterisk is a way of denoting it in programming.
\end{tsremark}

\begin{codestore}[ex7]
  \begin{equation}
    \xinterval{2, 10 * 1, 15 * -3, 19}
  \end{equation}
\end{codestore}
\tsdemo{ex7}

To combine this macro with the keys of \cs{interval}, the latter must be specified as a list delimited by semicolons (the comma being already used to separate the different keys). Here is an example to clarify this point.

\begin{codestore}[ex8]
  \begin{equation}
    \xinterval[open right, scaled=Big; open left, scaled=Big]{2, 10 * 1, 15}
  \end{equation}
\end{codestore}
\tsdemo{ex8}

\subsection{Union}

\begin{function}{\uinterval}
  \begin{syntax}
    \oarg{keys}\marg{\textbar-list}
  \end{syntax}
  Similarly, interval union is typeset using the vertical bar as a delimiter.
\end{function}

\begin{tsremark}[Mnemonic:]
  The \enquote{u} at the beginning of the macro name evokes the union symbol $\cup$, while the vertical bar is a way in programming to denote the logical \textit{or}.
\end{tsremark}

\begin{codestore}[ex9]
  \begin{equation}
    \uinterval{2, 10 | 1, 15}
  \end{equation}
\end{codestore}
\tsdemo{ex9}

\subsection{Intersection}

\begin{function}{\ninterval}
  \begin{syntax}
    \oarg{keys}\marg{\&-list}
  \end{syntax}
  Again, interval intersection is typeset using the ampersand as a delimiter. The use of keys is also identical to \cs{xinterval} and \cs{uinterval}.
\end{function}

\begin{tsremark}[Mnemonic:]
  The \enquote{n} evokes the intersection symbol $\cap$, with the ampersand denoting the logical \emph{and} in programming.
\end{tsremark}

\begin{codestore}[ex10]
  \begin{equation}
    \ninterval{3, 20 & \pi, e^3}
  \end{equation}
\end{codestore}
\tsdemo{ex10}

\section{Inequalities}

\begin{function}{\ineq}
  \begin{syntax}
    \sarg\oarg{keys}\marg{list of bounds}\oarg{variable}
  \end{syntax}
  The composition of inequalities is quite similar to that of intervals, with a few differences. The keys \texttt{open right}, \texttt{open left} and \texttt{open} are also available to make it easy to write strict or non-strict inequalities. In addition to the bounds, you can also specify the variable, which defaults to $x$. Finally, the starred variant of the macro uses the alternative symbols from \pkg{amssymb} for non-strict inequalities.

\end{function}

\begin{codestore}[ex11]
  \begin{gather}
    \ineq{a, b}\\
    \ineq*{a, b}[y]\\
    \ineq[open right]{a, b}\\
    \ineq[open left]{a, b}\\
    \ineq[open]{a, b}
  \end{gather}
\end{codestore}
\tsdemo{ex11}

\addcontentsline{toc}{section}{Changelog}

\begin{changelog}[sectioncmd=\section*, changes=Initial version]
\shortversion{v=1.0.0, date=2025-11-16}
\end{changelog}

\PrintIndex

\end{documentation}

\end{document}
\endinput
%%
%% End of file `intervalx-doc-en.tex'.
